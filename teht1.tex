
\chapter{Tehtävä 1 \label{chap:Teht=0000E4v=0000E4-1}}

Viittaaminen lukuun \ref{chap:Teht=0000E4v=0000E4-1}, toiseen lukuun
\ref{chap:Teht=0000E4v=0000E4-2}, alilukuun \ref{Alaotsikko}, tätä
alempaan lukuun \ref{Alempiotsikko}, alimpaan lukuun \ref{Alinotsikko},
kuvaan \ref{Kuva esimerkki} ja tauluun \ref{Tauluesimerkki}.

Kuva liitetään seuraavasti. ShareLaTeXin autocomplete rakentaa
koko begin-end blockin yleensä puolestasi.

\begin{figure}
\centering \includegraphics[width=0.5\textwidth]{kuvat/turun-yliopisto}
\caption{Kuvan otsikko}
\label{Kuva esimerkki} 
\end{figure}

Taulukkoja tehdään seuraavasti.

\begin{table}
\begin{centering}
\caption{Taulukon otsikko tulee taulun yläpuolelle}
\begin{tabular}{l|c|r|}
Taulun  & elementit  & erotetaan \tabularnewline
\hline 
toisistaan  & et-merkillä  & \tabularnewline
soluja voi myös  &  & jättää tyhjäksi \tabularnewline
\end{tabular}
\par\end{centering}
\centering{}\label{Tauluesimerkki}
\end{table}

Kirjallisuusviitteet lisätään bib-muodossa bibliografia
tiedostoon ja niihin viitataan niiden ID:llä, joka on bib-muodon
ensimmäinen kenttä \cite{crawley2007write}.

\section{Alaotsikko}

\label{Alaotsikko}

Esimerkki viittaamisesta, jossa myös cite komennon tagi löytyy
Bibliografia.bib tiedostosta \cite{puasuareanu2009survey}.

\subsection{Alempiotsikko}

\label{Alempiotsikko}

Lorem ipsum dolor sit amet, consectetur adipiscing elit. Etiam eget
tellus porttitor, tempus lacus non, pellentesque ligula. Donec sit
amet erat condimentum, feugiat mi accumsan, euismod quam.

Mauris laoreet maximus aliquet. Mauris at gravida elit. Ut nec lobortis
elit. Sed lacinia nisi in ex sollicitudin, ac consequat lacus imperdiet.
Etiam et velit eu lacus maximus faucibus.

\subsubsection{Alinotsikko, joka ei näy sisällysluettelossa}

\label{Alinotsikko}
\begin{javacode}
// java-koodin voi kirjoittaa tallaiseen ymparistoon
// jolloin koodi syntaksivarjataan automaattisesti
// lyx ei tue tassa skandeja, mutta puhdas (xe)latex tukee
public class HelloWorld {
  public static void main(String[] args) {
    // Prints "Hello, World" to the terminal window.
    System.out.println("Hello, World");
  }
}
\end{javacode}

\paragraph{Otsikko tekstissä, joka ei näy sisällysluettelossa}

Mauris laoreet maximus aliquet. Mauris at gravida elit. Ut nec lobortis
elit. Sed lacinia nisi in ex sollicitudin, ac consequat lacus imperdiet.
Etiam et velit eu lacus maximus faucibus. Vestibulum ante ipsum primis
in faucibus orci luctus et ultrices posuere cubilia Curae; Donec vulputate
tellus ullamcorper odio sodales, non scelerisque neque eleifend. 
