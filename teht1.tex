
\chapter{Tehtävä 1 \label{chap:Teht=0000E4v=0000E4-1}}

Tehtävänä on vähentää OOMKitin DemoApp4:n riippuvuutta OOMKittiin.

\section{Kohta (a)}
\label{Kohta (a)}

Näitä metodeja voisi käyttää tehtävän kuvaamissa tilanteissa, kunhan on olemassa GraphicsContext canvas johon vaikuttaa. Demo käyttää SimpleCanvas-luokkaa, joka periaattessa vain helpottaa värien vaihtoa (setFill() ja setStroke()) ja muuta vastaavaa ylläpitoa.

\subsubsection{Piirtävät luokat}
\label{Piirtävät luokat}
\begin{javacode}
// piirtää pallukan pisteeseen p värillä väri
  public void piirräPallukka(Point p, Color väri, double koko, boolean täytetty,
      GraphicsContext canvas) {
    if (täytetty){
      if (debuggaus) System.out.println("Piirretään täytetty " + (koko < 15 ?
          "pieni pallukka" : (int)koko+" cm - säteinen ympyrä") + " pisteeseen " + p);
      setFill(väri);
      canvas.fillOval(p.x - koko / 2, p.y - koko / 2, koko, koko);
    } else {
      if (debuggaus) System.out.println("Piirretään ontto " + (koko < 15 ?
         "pieni pallukka" : (int)koko+" cm - säteinen ympyrä") + " pisteeseen " + p);
      canvas.setLineWidth(2);
      setStroke(väri);
      canvas.strokeOval(p.x - koko / 2, p.y - koko / 2, koko, koko);
    }
  }
  // piirtää viivaan pisteestä alku pisteeseen lopppu värillä väri
  public void piirräViiva(Point alku, Point loppu, Color väri, GraphicsContext canvas) {
    if (debuggaus) System.out.println("Piirretään viiva pisteestä " + alku +
        " pisteeseen " + loppu);
    canvas.setLineWidth(2);
    setStroke(väri);
    canvas.strokeLine(alku.x, alku.y, loppu.x, loppu.y);
  }
  // piirtää viivaan pisteestä alku pisteeseen lopppu värillä väri
  public void piirräLaatikko(Point vasenYlä, Point oikeaAla, Color väri, boolean täytetty,
      GraphicsContext canvas) {
    if (täytetty) {
      if (debuggaus) System.out.println("Piirretään täytetty laatikko pisteiden " + vasenYlä
          + " ja " + oikeaAla + " välille");
      setFill(väri);
      canvas.fillRect(vasenYlä.x, vasenYlä.y, oikeaAla.sub(vasenYlä).x, 
          oikeaAla.sub(vasenYlä).y);
    } else {
      if (debuggaus) System.out.println("Piirretään ontto laatikko pisteiden " + vasenYlä
          + " ja " + oikeaAla + " välille");
      canvas.setLineWidth(2);
      setStroke(väri);
      canvas.strokeRect(vasenYlä.x, vasenYlä.y, oikeaAla.sub(vasenYlä).x,
          oikeaAla.sub(vasenYlä).y);
    }
  }
  public void piirräKolmio(Point keski, double koko, GraphicsContext canvas) {
    Point piste1 = kaarelta(keski, Math.PI * 1 / 6, koko / 2 / 37 * 50);
    Point piste2 = kaarelta(keski, Math.PI * 5 / 6, koko / 2 / 37 * 50);
    Point piste3 = kaarelta(keski, Math.PI * 9 / 6, koko / 2 / 37 * 50);

    int siirtymäY = keski.y - (piste2.y + piste3.y) / 2;
    piste1 = piste1.add(0, siirtymäY);
    piste2 = piste2.add(0, siirtymäY);
    piste3 = piste3.add(0, siirtymäY);

    canvas.fillPolygon(new double[]{piste1.x, piste2.x, piste3.x}, new double[]{piste1.y,
        piste2.y, piste3.y}, 3);
  }
  public void updateContent() {
    if (kulma > 0.5) debuggaus = false;
    kulma += 0.1;
  }
  // palauttaa pistettä keski ympäröivän säde-säteisen ympyrän kaarelta pisteen kulmasta
  protected static Point kaarelta(Point keski, double kaariKulma, double säde) {
    return keski.add(
        (int) (säde * Math.cos(kaariKulma)),
        (int) (säde * Math.sin(kaariKulma))
    );
  }
  protected static String fyysisesti(Point p) {
    return "kohtaan, joka on " + p.x + " cm kohteen vasemmasta reunasta ja " + p.y +
        " cm yläreunasta.";
  }
  protected void drawBackgroundContent(GraphicsContext canvas) {
    setFill(CoreColor.Black);
    canvas.fillRect(0, 0, getWidth(), getHeight());
  }
  protected void drawForegroundContent(GraphicsContext canvas) {
    Point keski1 = new Point(100, 100);
    Point keski2 = new Point(300, 150);
    Point keski3 = new Point(150, 350);

    {
      Point keski = keski1;
      int kulmia = 5;
      int kulmaHyppy = 2;
      double kulmaLisäys = kulma;
      double koko = 32;

      if (debuggaus) System.out.println("Seuraavilla ohjeilla piirretään 5-kulmio kohtaan "
          + fyysisesti(keski));
      for (double i = 0; i < kulmia; i++) {
        Point p1 = kaarelta(keski, kulmaLisäys + i / kulmia * 2 * Math.PI, koko);
        Point p2 = kaarelta(keski, kulmaLisäys + (i + kulmaHyppy) / kulmia * 2 * Math.PI,
            koko);
        piirräViiva(p1, p2, CoreColor.Red, canvas);
        piirräPallukka(p1, CoreColor.LightRed, 10, true, canvas);
        piirräPallukka(p2, CoreColor.LightRed, 10, true, canvas);
      }
    }
    {
      Point keski = keski2;
      int kulmia = 7;
      int kulmaHyppy = 1;
      double kulmaLisäys = -kulma / 2;
      double koko = 80;

      if (debuggaus) System.out.println("Seuraavilla ohjeilla piirretään 7-kulmio kohtaan "
          + fyysisesti(keski));
      for (double i = 0; i < kulmia; i++) {
        Point p1 = kaarelta(keski, kulmaLisäys + i / kulmia * 2 * Math.PI, koko);
        Point p2 = kaarelta(keski, kulmaLisäys + (i + kulmaHyppy) / kulmia * 2 * Math.PI,
            koko);
        piirräViiva(p1, p2, CoreColor.Red, canvas);
        piirräPallukka(p1, CoreColor.LightRed, 10, true, canvas);
        piirräPallukka(p2, CoreColor.LightRed, 10, true, canvas);
      }
    }
    {
      Point sijainti = keski3.add((int) (Math.cos(kulma / 2) * 180 + 100), 
          (int) (Math.sin(kulma / 2) * 40));
      int koko = 80;
      Color väri = CoreColor.Orange;

      if (debuggaus) System.out.println("Seuraavilla ohjeilla piirretään"+
          " 'Clavicula Nox' kohtaan " + fyysisesti(sijainti));
      piirräPallukka(sijainti, väri, koko, false, canvas);
      piirräPallukka(sijainti.add(0, koko / 2), väri, koko * 2 / 3, false, canvas);
      piirräLaatikko(sijainti.add(-koko / 2, -koko / 2 - 5), sijainti.add(koko / 2, 0),
          CoreColor.Black, true, canvas);
      piirräViiva(sijainti.add(0, koko * 2), sijainti.add(0, -koko / 4), väri, canvas);
    }
  }
\end{javacode}

\section{Kohta (b)}
\label{Kohta (b)}

Näistä metodeista ohjeiden tulostus ei olisi vaikeaa, sillä se tekee sen jo dubuggaus-booleanin ollessa true. Tietysti joitain tekstejä tulisi muokata "ohjemaisemmiksi" mutta muuten en näe ongelmaa.


Tekijänä Santeri Loitomaa (516587)
