
\chapter{Tehtävä 1 \label{chap:Teht=0000E4v=0000E4-1}}

Tehtävässä tutustutaan oom-kit -kehykseen ja siitä löytyviin luokkiin

\section{fi.utu.oomkit -luokat}

\label{fi.utu.oomkit -luokat}

fi.utu.oomkit.gui -paketista löytyy luokat:

\begin{javacode}
public interface Console {}
class DefaultDialogFactory implements DialogFactory {}
public interface DialogFactory {}
class FXConsole extends OutputStream implements Console {}
public interface MainWindow extends WindowContent {}
public abstract class MainWindowBase implements MainWindow {}
class MergedStream extends OutputStream {}
public abstract class OOMApp extends Application {}
public class ReactiveCanvas<X> extends SimpleCanvas implements Observer<X> {}
public class ReactiveLabel<T> extends Label implements Observer<T> {
	public interface LabelHandler<T> {}
}
public class SimpleCanvas extends Canvas {}
class StreamWrapper extends OutputStream {}
public interface WindowContent {}
\end{javacode}

Joista kaikki ovat julkisia paketin sisällä, mutta paketin ulkopuolella
näkyvistä on jätetty luokat DefaultDialogFactory, FXConsole, MergedStream ja
StreamWrapper. fi.utu.oomkit -paketista löytyy luokat:

\begin{javacode}
public final class AppConfiguration {}
public interface AppLogic extends KeyHandler, Scheduled {}
\end{javacode}

Joista kaikki ovat julkisia luokkia niin paketin sisällä kuin ulkopuolella.

\section{OOMApp -luokka sovelluksen ikkunan luomiseksi}

\label{OOMApp -luokka sovelluksen ikkunan luomiseksi}
 
OOMApp -luokka peritään ja sille kirjoitetaan generateMainWindow -metodi.

GenerateMainWindow -metodilla määritellään mikä on ohjelman (=app) nimi ja
kuinka ison alueen ohjelma käyttää. Metodin tulee palauttaa MainWindow
-alaluokka. OOMApp -luokan käyttöä on esitetty DemoApp1 -tutoriaalissa
seuraavasti:

\begin{javacode}
public class DemoApp1 extends OOMApp {
    // alustaa pelilogiikan
    final static LaatikkoGameLogic gameLogic = new LaatikkoGameLogic();

    // kytkee piirtopinnan käyttöliittymään
    @Override
    protected MainWindow generateMainWindow(String appName, double width,
    double height) { return new SimpleMainWindow(appName, width, height) {
            @Override public SimpleCanvas mainContent() {
                return gameLogic.piirtoPinta;
            }
        };
    }
}
\end{javacode}

EmptyApp -luokassa generateMainWindow -metodille on tehty mielekäs
esimerkki-implementaatio, jolloin sitä voi käyttää sellaisenaan muodostaessa
ohjelman ikkunaa. Siitä löytyy valmiiksi ajettava main-metodi.

\section{Ohjelman ikkunan sisällön hallitseminen}

\label{}

Ikkuna luodaan käyttämällä luokkaa MainWindow. Huomionarvoista on, että
MainWindow -olion sisältö koostuu ylä- ja alapalkista, sekä
Canvas-piirtoalueesta. Oletuksena nämä ovat tyhjiä, jolloin oman sisällön voi
luoda ylikirjoittamalla metodit topBarContent(), mainContent(),
bottomBarContent(). 

\begin{javacode}
public class EmptyApp extends OOMApp {
	@Override
	protected MainWindow generateMainWindow(){
		return new SimpleMainWindow(appName, width, height) {
			
			@Override
			public List<Node> topBarContent() { 
				//return yläpalkin sisältö
			}
		
			@Override
			public Canvas mainContent() { 
				//return pääsisältö
			}
		
			@Override
			public List<Node> bottomBarContent() { 
				//return alapalkin sisältö
			}
		}
	}
}
\end{javacode}

Ylä- ja alapalkkiin voidaan lisätä mitä tahansa graafisia solmuja, jotka voivat
olla erilaisia graafisia käyttöliittymäkomponentteja. Ylikirjoittamalla
topBarContent() -metodin voidaan sijoittaa käyttöliittymään lista mitä tahansa
elementtejä kuten esimerkiksi luokkien Canvas, ImageView, Shape tai Button
luokkien olio. Yläpalkin lisäämisessä voitaisiin käyttää esimerkiksi seuraavaa
metodia:

\begin{javacode}
@Override
public List<Node> topBarContent() {
	return Arrays.asList(
		new Button("File"),
		new Button("Edit"),
		new Button("Window"),
		new Button("Help")
	);
}
\end{javacode}

\section{Ohjelmanaikaiset tapahtumat}

\label{D}

OOMApp -luokan olion konstruktorilla pystytään luomaan useanlaisia logiikoita
ohjelman suoritukseksi. Oma logiikka voidaan lisätä käyttämällä OOMApp
konstruktoria, johon syötetään ohjelmalogiikka. Ohjelmalogiikkaluokka peritään
AppLogic -luokalta. Seuraavassa esimerkissä on esitetty, kuinka MyLogic
-luokalla voidaan hallita ohjelmalogiikkaa niin, että asiakkaan koodi
suoritetaan joka 25ms.

\begin{javacode}
public class MyApp extends OOMApp {
	public MyApp() {
		super(new MyLogic());
	}
}

public class MyLogic extends AppLogic {
	@override
	AppConfiguration configuration() {
		return new AppConfiguration(25, "Asiakkaan ohjelma", true);
	}
    
	@override
	void tick() {
		//asiakkaan koodi
	}
}
\end{javacode}

- Pasi Toivanen (517487)
