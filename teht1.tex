
\chapter{Tehtävä 1 \label{chap:Teht=0000E4v=0000E4-1}}

Tehtävässä tutustutaan oom-kit -kehykseen ja siitä löytyviin luokkiin

\section{Kohta A}

\label{Kohta A}

Gui -paketista löytyy luokat:

\begin{javacode}
public interface Console {}
class DefaultDialogFactory implements DialogFactory {}
public interface DialogFactory {}
class FXConsole extends OutputStream implements Console {}
public interface MainWindow extends WindowContent {}
public abstract class MainWindowBase implements MainWindow {}
class MergedStream extends OutputStream {}
public abstract class OOMApp extends Application {}
public class ReactiveCanvas<X> extends SimpleCanvas implements Observer<X> {}
public class ReactiveLabel<T> extends Label implements Observer<T> {
	public interface LabelHandler<T> {}
}
public class SimpleCanvas extends Canvas {}
class StreamWrapper extends OutputStream {}
public interface WindowContent {}
\end{javacode}

Joista kaikki ovat julkisia paketin sisällä, mutta paketin ulkopuolella
näkyvistä on jätetty luokat DefaultDialogFactory, FXConsole, MergedStream ja
StreamWrapper. Oom-kit -paketista löytyy luokat:

\begin{javacode}
public final class AppConfiguration {}
public interface AppLogic extends KeyHandler, Scheduled {}
\end{javacode}

Joista kaikki ovat julkisia luokkia niin paketin sisällä kuin ulkopuolella.

\section{Kohta B}

\label{Kohta B}
 
OOMApp -luokka peritään ja sille kirjoitetaan generateMainWindow -metodi.

GenerateMainWindow -metodilla määritellään mikä on ohjelman (=app) nimi ja
kuinka ison alueen ohjelma käyttää. Metodin tulee palauttaa MainWindow
-alaluokka. OOMApp -luokan käyttöä on esitetty DemoApp1 -tutoriaalissa
seuraavasti:

\begin{javacode}
public class DemoApp1 extends OOMApp {
    // alustaa pelilogiikan
    final static LaatikkoGameLogic gameLogic = new LaatikkoGameLogic();

    // kytkee piirtopinnan käyttöliittymään
    @Override
    protected MainWindow generateMainWindow(String appName, double width, double height) {
        return new SimpleMainWindow(appName, width, height) {
            @Override public SimpleCanvas mainContent() {
                return gameLogic.piirtoPinta;
            }
        };
    }
}
\end{javacode}

EmptyApp -luokassa generateMainWindow -metodille on tehty mielekäs
esimerkki-implementaatio, jolloin sitä voi käyttää sellaisenaan muodostaessa
ohjelman ikkunaa. Siitä löytyy valmiiksi ajettava main-metodi.

\section{Kohta C}

\label{Kohta C}

KESKEN:
Kuvaile lyhyesti tähän tarvit- tavat vaiheet. Mitä näkyviä osakomponentteja ikkunassa on ja miten niitä voi itse muokata. Kuvaile lyhyesti
tähän liittyvät metodit.
