
\chapter{Tehtävä 4 \label{chap:Teht=0000E4v=0000E4-4}}

Tehtävän tarkoituksena oli demostroida javan metodikutsurakenteen toimintaaa ja mahdollisia rajoitteita.
Tehtävässä on luotu maailma jossa on Hiiri, Kissa ja Hipsteri olioita. Kun kaksi oliota tapaavat tulostuu näytölle testi joka määräytyy olioiden luokan mukaan.

\section{Kuormitus(a)}

\label{Kuormitus}

Loin jokaiselle mahdolliselle olion yhdistelmälle kohtaa-metodin. 
Kaikilla metodeilla on täysin sama nimi ja vain parametrit muuttuvat.

\label{Olio-luokat}
\begin{javacode}
public class Hipsteri extends Olio {
	public Hipsteri() {}
	
	public void kohtaa(Hipsteri h) {
		System.out.println("Hipsteri kohtaa hipsterin");
	}
    public void kohtaa(Kissa k) {
    	System.out.println("Hipsteri kohtaa eläimen");
	}
    public void kohtaa(Hiiri h) {
    	System.out.println("Hipsteri kohtaa eläimen");
	}
}
public class Kissa extends Olio{
	public Kissa() {}
	
	public void kohtaa(Hipsteri h) {
		System.out.println("Hipsteri kohtaa eläimen");
	}
	public void kohtaa(Kissa k) {
		System.out.println("Kissoja syntyy mahdollisesti nyt");
	}
	public void kohtaa(Hiiri h) {
		System.out.println("Hiiri syödään");
	}
}
public class Hiiri extends Olio {
	public Hiiri() {}
	
	public void kohtaa(Hipsteri h) {
		System.out.println("Hipsteri kohtaa eläimen");
	}
	public void kohtaa(Hiiri h) {
		System.out.println("Hiiriä syntyy mahdollisesti nyt");
	}
	public void kohtaa(Kissa k) {
		System.out.println("Hiiri syödään");
	}
}
\end{javacode}
Ja sitten yksin kertainen tarkistus että koodi toimii niin kuin ajattelin.

\label{Koodin testaus}
\begin{javacode}
public static void main(String[]args) {
    	Hipsteri harri = new Hipsteri();
    	Hipsteri tiina = new Hipsteri();
    	Hiiri pip = new Hiiri();
    	Kissa misu = new Kissa();
    	harri.kohtaa(misu);
    	pip.kohtaa(harri);
    	misu.kohtaa(pip);
    	tiina.kohtaa(harri);
      }
\end{javacode}

\section{Yliluokan käyttö(b)}

\label{Yliluokan käyttö}
 Seuraavaksi laitoin olioita taulukkoon ja valitsin kahdesta taulukon indeksista olion ja laitoin ne kohtaamaan.
 \begin{javacode}
public class Maailma {
	public Olio[] map;
	
    public Maailma() {
    	map = new Olio[10];
    }
    public void lisää(Olio o,int i) {
    	map[i]=o;
    }
    public void poista(int i) {
    	map[i]=null;
    }
    public Olio get(int i) {
    	return map[i];
    }
     public static void main(String[]args) {
    	
    	Maailma m = new Maailma();
    	Hipsteri harri = new Hipsteri();
    	Hipsteri tiina = new Hipsteri();
    	Hiiri pip = new Hiiri();
    	Hiiri pup = new Hiiri();
    	Kissa misu = new Kissa();
    	Kissa kisu = new Kissa();
    	
    	m.lisää(harri, 0);
    	m.lisää(tiina, 1);
    	m.lisää(pip, 2);
    	m.lisää(misu, 3);
    	m.lisää(pup, 4);
    	m.lisää(kisu, 5);
  
    	m.get(0).kohtaa(m.get(1));
    	m.get(0).kohtaa(m.get(2));
    	m.get(0).kohtaa(m.get(3));
    	m.get(2).kohtaa(m.get(4));
    	m.get(2).kohtaa(m.get(3));
    	m.get(3).kohtaa(m.get(3));
    	m.get(3).kohtaa(m.get(1));
    }
    
\end{javacode}

Tässä tilanteessa kohtaa-metodit eivät toimi sellaisenaan sillä parametri on
olio-tyyppisenä. Tässä tilanteessa voitaisiin dynaamisesti muuttaa parametrin tyyppiä, mutta
tämä toimii vain jos parametrin tyyppi on ennestään tiedossa. Itse tekisin yläluokkaan yhden metodin joka tarkistaa molempien 
olioden tyypit. 

\begin{javacode}
public void kohtaa(Olio o) {
		if(this instanceof Kissa) {
			if(o instanceof Kissa) {
				System.out.println("Kissoja syntyy mahdollisesti nyt");
			}else if(o instanceof Hiiri) {
				System.out.println("Kissa syö hiiren");
		}else if(o instanceof Hipsteri) {
			System.out.println("Hipsteri kohtaa eläimen");
	    }
	} if(this instanceof Hiiri) {
		if(o instanceof Kissa) {
			System.out.println("Kissa syö hiiren");
		}else if(o instanceof Hiiri) {
			System.out.println("Hiiriä syntyy mahdollisesti nyt");
	}else if(o instanceof Hipsteri) {
		System.out.println("Hipsteri kohtaa eläimen");
}
	} if(this instanceof Hipsteri) {
		if(o instanceof Hipsteri) {
			System.out.println("Hipsteri kohtaa hipsterin");
		}else {
			System.out.println("Hipsteri kohtaa eläimen");
		}
		
		}
}
\end{javacode}



\section{Vierailija(c)}
\label{Vierailija}
\begin{javacode}
interface Visitor{
	void visit(Kissa k,Olio o);
	void visit(Hiiri h,Olio o);
	void visit(Hipsteri h,Olio o);
	
}

public class OlioVisitor implements Visitor {

	@Override
	public void visit(Kissa k, Olio o) {
			if(o instanceof Kissa) {
				System.out.println("Kissoja syntyy mahdollisesti nyt");
			}else if(o instanceof Hiiri) {
				System.out.println("Kissa syö hiiren");
		}else if(o instanceof Hipsteri) {
			System.out.println("Hipsteri kohtaa eläimen");
	    }
	}

	@Override
	public void visit(Hiiri h, Olio o) {
		if(o instanceof Kissa) {
			System.out.println("Kissa syö hiiren");
		}else if(o instanceof Hiiri) {
			System.out.println("Hiiriä syntyy mahdollisesti nyt");
	}else if(o instanceof Hipsteri) {
		System.out.println("Hipsteri kohtaa eläimen");
    }
		
	}

	@Override
	public void visit(Hipsteri h, Olio o) {
		if(o instanceof Kissa) {
			System.out.println("Hipsteri kohtaa eläimen");
		}else if(o instanceof Hiiri) {
			System.out.println("Hipsteri kohtaa eläimen");
	}else if(o instanceof Hipsteri) {
		System.out.println("Hipsteri kohtaa hipsterin");
    }
		
	}

}


\end{javacode}
ja jokaiselle oliolle vastaanottaja metodi
\begin{javacode}
public void Accept(OlioVisitor visitor,Olio o) {
    	visitor.visit(this,o); 
    	} 
	
\end{javacode}





Tekijänä Janina Kuosmanen (516580)

\label{endofpages}
