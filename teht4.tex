
\chapter{Tehtävä 4 \label{chap:Teht=0000E4v=0000E4-4}}

Tehtävän tarkoituksena on luoda mallinnus OOM-kurssin aiheuttaman tuskan minimoivasta metodista.

\section{Opiskelija-luokat (kohdat a)}

\label{Opiskelija-luokat}

Kohdassa a tuli määrittää ja toteuttaa luokat TavallinenOpiskelija ja ValeasuOpiskelija. Tulin lopputulokseen että TavallinenOpiskelija on vain Opiskelija-rajapinnan toteutus mutta ValeasuOpiskelija on nimenomaan valeasu, jonka ainoa arvo on TavallinenOpiskelija ja kaikki metodit suunnataan suoraan tälle opiskelijalle.

\subsubsection{TavallinenOpiskelija-luokka}
\label{TavallinenOpiskelija-luokka}
\begin{javacode}
public class TavallinenOpiskelija implements Opiskelija {
  String nimi;
  int opNumero;
  OOMTilanne oomTilanne;
  public TavallinenOpiskelija(String nimi, int opNumero, boolean hereillä) {
    this.nimi = nimi;
    this.opNumero = opNumero;
    this.oomTilanne = new OOMTilanne();
    oomTilanne.hereillä = hereillä;
  }
  /**
   * @.pre true
   * @.post RESULT == (Palauttaa tilanneolion)
   */
  @Override
  public OOMTilanne annaOOMTilanne() {
    return oomTilanne;
  }
  /**
   * Asettaa opiskelijan nimen ja op. numeron
   * 
   * @.pre nimi != null && opNumero>0
   * @.post (nimi & op.numero asetettu)
   **/
  @Override
  public void asetaNimiJaOpNumero(String nimi, int opNumero) {
    this.nimi = nimi;
    this.opNumero = opNumero;
  }
  /**
   * @.pre true
   * @.post annaOOMTilanne().ilmoittautunut == true && annaOOMTilanne().hereillä
   *        == false && annaOOMTilanne().luennolla == false &&
   *        (OLD(annaOOMTilanne().ilmoittautunut) || Maailma.tuskanMäärä ==
   *        OLD(Maailma.tuskanMäärä) + 1000)
   */
  @Override
  public void ilmoittauduOOMKurssille() {
    if(!oomTilanne.ilmoittautunut) {
      oomTilanne.ilmoittautunut = true;
      Maailma.lisääTuskaa(1000);
    }
  }
  /**
   * @.pre annaOOMTilanne().ilmoittautunut == true && annaOOMTilanne().hereillä ==
   *       false
   * @.post Maailma.tuskanMäärä == OLD(Maailma.tuskanMäärä) + 10 &&
   *        annaOOMTilanne().luennolla == true
   */
  @Override
  public void osallistuLuennolle() {
    oomTilanne.luennolla = true;
    Maailma.lisääTuskaa(10);
    if(oomTilanne.hereillä) {
      Maailma.lisääTuskaa(90);
    }
  }
  /**
   * @.pre 0 ≤ aikaaLuennonAlusta < 90
   * @.post annaOOMTilanne().hereillä == true
   */
  @Override
  public void herää(int aikaaLuennonAlusta) {
    oomTilanne.hereillä = true;
  }
  /**
   * @.pre annaOOMTilanne().luennolla == true
   * @.post annaOOMTilanne().luennolla == false && annaOOMTilanne().hereillä ==
   *        false
   */
  @Override
  public void poistuLuennolta() {
    oomTilanne.luennolla = false;
    oomTilanne.hereillä = false;
  }
  /**
   * @.pre annaOOMTilanne().ilmoittautunut == true && annaOOMTilanne().hereillä ==
   *       true
   * @.post Maailma.tuskanMäärä == OLD(tuskanMäärä) + 90 -
   *        hereilläAlkaenMinuutista && annaOOMTilanne().hereillä == true
   */
  @Override
  public void vastaaKysymykseen(int aikaaLuennonAlusta) {
    herää(aikaaLuennonAlusta);
    Maailma.lisääTuskaa(90-aikaaLuennonAlusta);
  }
}
\end{javacode}

\subsubsection{ValeasuOpiskelija-luokka}
\label{ValeasuOpiskelija-luokka}
\begin{javacode}
public class ValeasuOpiskelija implements Opiskelija {
  Opiskelija opiskelija;
  public ValeasuOpiskelija(Opiskelija opiskelija) {
    this.opiskelija = opiskelija;
  }
  /**
   * @.pre true
   * @.post RESULT == (Palauttaa tilanneolion)
   */
  @Override
  public OOMTilanne annaOOMTilanne() {
    return opiskelija.annaOOMTilanne();
  }
  /**
   * Asettaa opiskelijan nimen ja op. numeron
   * 
   * @.pre nimi != null && opNumero>0
   * @.post (nimi & op.numero asetettu)
   **/
  @Override
  public void asetaNimiJaOpNumero(String nimi, int opNumero) {
    opiskelija.asetaNimiJaOpNumero(nimi, opNumero);
  }
  /**
   * @.pre true
   * @.post annaOOMTilanne().ilmoittautunut == true && annaOOMTilanne().hereillä
   *        == false && annaOOMTilanne().luennolla == false &&
   *        (OLD(annaOOMTilanne().ilmoittautunut) || Maailma.tuskanMäärä ==
   *        OLD(Maailma.tuskanMäärä) + 1000)
   */
  @Override
  public void ilmoittauduOOMKurssille() {
    opiskelija.ilmoittauduOOMKurssille();
  }
  /**
   * @.pre annaOOMTilanne().ilmoittautunut == true && annaOOMTilanne().hereillä ==
   *       false
   * @.post Maailma.tuskanMäärä == OLD(Maailma.tuskanMäärä) + 10 &&
   *        annaOOMTilanne().luennolla == true
   */
  @Override
  public void osallistuLuennolle() {
    opiskelija.osallistuLuennolle();
  }
  /**
   * @.pre 0 ≤ aikaaLuennonAlusta < 90
   * @.post annaOOMTilanne().hereillä == true
   */
  @Override
  public void herää(int aikaaLuennonAlusta) {
    opiskelija.herää(aikaaLuennonAlusta);
  }
  /**
   * @.pre annaOOMTilanne().luennolla == true
   * @.post annaOOMTilanne().luennolla == false && annaOOMTilanne().hereillä ==
   *        false
   */
  @Override
  public void poistuLuennolta() {
    opiskelija.poistuLuennolta();
  }
  /**
   * @.pre annaOOMTilanne().ilmoittautunut == true && annaOOMTilanne().hereillä ==
   *       true
   * @.post Maailma.tuskanMäärä == OLD(tuskanMäärä) + 90 -
   *        hereilläAlkaenMinuutista && annaOOMTilanne().hereillä == true
   */
  @Override
  public void vastaaKysymykseen(int aikaaLuennonAlusta) {
    opiskelija.vastaaKysymykseen(aikaaLuennonAlusta);
  }
}
\end{javacode}



\section{OpiskelijaAllas-luokka (kohta c)}

\label{OpiskelijaAllas-luokka}

Kohdassa c pyydetään toteuttamaan luokka, jonka avulla kierrätetään opiskelijoita, jottei kaikkien tarvitse ilmoittautua tälle tuskaa tuottavalle kurssille.

\subsubsection{OpiskelijaAllas-luokka}
\label{OpiskelijaAllas-luokka}
\begin{javacode}
public class OpiskelijaAllas {
  private static Opiskelija[] opiskelijat = new Opiskelija[0];  
  /**
   * Metodi joka luo uusia opiskelijoita vain tarpeen vaatiessa.
   * @param uhriMäärä
   * @param vuosi
   */
  public static void hankiOpiskelijat(int uhriMäärä, int vuosi) {
    Opiskelija[] uusiAllas = new Opiskelija[uhriMäärä];
    if(uhriMäärä > opiskelijat.length) {
      for(int i = 0; i < opiskelijat.length; i++) {
        uusiAllas[i] = opiskelijat[i];
      }
      for(int i = opiskelijat.length; i < uhriMäärä; i++) {
        uusiAllas[i] = new ValeasuOpiskelija(new TavallinenOpiskelija("Uniikki Lumihiutale", 1234567 + i, false));
      }
    }
    else {
      for(int i = 0; i < uhriMäärä; i++) {
        uusiAllas[i] = opiskelijat[i];
      }
    }
    for (int i = 0; i < uhriMäärä; i++) {
      uusiAllas[i].asetaNimiJaOpNumero("Uniikki Lumihiutale", 1234567 + i + vuosi);
    }
    opiskelijat = uusiAllas;
  }  
  /**
   * Metodi joka luo aina uudet opiskelijat.
   * @param uhriMäärä
   */
  public static void hankiOpiskelijat(int uhriMäärä) {
    Opiskelija[] uusiAllas = new Opiskelija[uhriMäärä];
    for (int i = 0; i<uhriMäärä; i++)
    uusiAllas[i] = new TavallinenOpiskelija("Uniikki Lumihiutale", 1234567+i, true);
    opiskelijat = uusiAllas;
  }
  /**
   * Palauttaa opiskelijat.
   * @return
   */
  public static Opiskelija[] opiskelijat() {
    return opiskelijat;
  }  
}
\end{javacode}



\section{Opiskelija testi (kohta b ja d)}

\label{Opiskelija testi}

Kohdassa b testataan luokkien eroavaisuuksia. Opiskelijoita 100 per vuosi.

Valeasulla pienenevä tuska.

\subsubsection{TavallinenOpiskelija-luokalla}
\label{TavallinenOpiskelija-luokalla}
\begin{javacode}
Tuska yhteensä: 117840
\end{javacode}

\subsubsection{ValeasuOpiskelija-luokalla}
\label{ValeasuOpiskelija-luokalla}
\begin{javacode}
Tuska yhteensä: 108840
\end{javacode}

Altailla pienenevä tuska.

\subsubsection{Ilman altaita}
\label{Ilman altaita}
\begin{javacode}
Vuoden 1 tuska: 108840
Vuoden 2 tuska: 108840
Vuoden 3 tuska: 108840
Vuoden 4 tuska: 108840
Vuoden 5 tuska: 108840
Tuska yhteensä: 544200
\end{javacode}

\subsubsection{Altailla}
\label{Altailla}
\begin{javacode}
Vuoden 1 tuska: 108840
Vuoden 2 tuska: 8840
Vuoden 3 tuska: 8840
Vuoden 4 tuska: 8840
Vuoden 5 tuska: 8840
Tuska yhteensä: 144200
\end{javacode}

ValeasuOpiskelija-allas luo vähiten tuskaa eli koodi toimii.

Tekijänä Santeri Loitomaa (516587)

\label{endofpages}
