
\chapter{Tehtävä 4 \label{chap:Teht=0000E4v=0000E4-1}}

Tehtävän tarkoituksena oli muodostaa oliorakenne, joka kuvaa Tetris-pelin
palasia. Tetris-palasen kuvaus on jaettu kahteen luokkaan: Piece, joka kuvaa
yhtä mielivaltaisen muotoista kappaletta – Shape, joka kuvaa Tetriksessä
käytettävien kappaleiden muotoja. Luokkatoteutus tehtiin seuraavanlaisella
java-koodilla:

\begin{javacode}
/**
 * 
 * Describes 4x4 piece in boolean values
 * @author Pasi Toivanen
 *
 */
public class Piece {

	private boolean[][] grid;
	private int gridSize;
	
//--Constructor (Muodostimet)
	/**
	 * 
	 * @param shape
	 */
	public Piece(Shape shape) {
		this.grid = shape.getGrid();
		this.gridSize = grid.length;
	}
	
	public void turnClockwise() {
		boolean[][] oldGrid = grid;
		grid = new boolean[gridSize][gridSize];
		for ( int i = 0; i < gridSize; i++) {
			for ( int j = 0; j < gridSize; j++) {
				grid[i][j] = oldGrid[gridSize-j-1][i];
			}
		}
	}
	
//--Getters (Havainnoijat)
	public String toString() {
		String result = "";
		for ( int i = 0; i < grid.length; i++) {
			String printRow = "";
			for ( int j = 0; j < grid[i].length; j++) {
				printRow += grid[i][j] ? "*" : " ";
			}
			result += printRow + "\n";
		}
		return result;
	}
	
	public boolean[][] getGrid() {
		return grid;
	}

	public int getGridSize() {
		return gridSize;
	}

//--Setters (Muutosoperaatiot)
	/**
	 * TODO:gridSize changes grid and expands or shrinks it accordingly
	 * @.pre gridSize > 0
	 * @.post this.gridSize.equals(gridSize)
	 * @param gridSize
	 */
	public void setGridSize(int gridSize) {
		this.gridSize = gridSize;
	}
	
	/**
	 * @.pre grid[i][j] = ( true || false ) for each i,j < gridSize
	 * @.post this.grid.equals(grid)
	 * @param grid TODO:this could be different size than this.grid
	 */
	public void setGrid(boolean[][] grid) {
		this.grid = grid;
	}

}

/**
 * 
 * Describes a shape in 4x4 boolean grid
 * @author Pasi Toivanen
 *
 */
enum Shape {
	SQUARE,
	LONG,
	PYRAMID,
	ZIGZAG;
	
	public boolean[][] getGrid() {
		boolean[][] result = new boolean[4][4];
		switch(this) {
			case SQUARE:
				result[1][1] = true; //
				result[1][2] = true; // **
				result[2][1] = true; // **
				result[2][2] = true; //
				break;
			case LONG:
				result[0][1] = true; //
				result[1][1] = true; //****
				result[2][1] = true; //
				result[3][1] = true; //
				break;
			case PYRAMID:
				result[0][2] = true; //
				result[1][2] = true; // *
				result[2][2] = true; //***
				result[1][1] = true; //
				break;
			case ZIGZAG:
				result[0][1] = true; //
				result[1][1] = true; //**
				result[1][2] = true; // **
				result[2][2] = true; //
				break;
		}
		return result;
	}
	
	/**
	 * @return TODO: different gridSize for pieces smaller than 4x4
	 */
	public int gridSize() {
		return 4;
	}
}
\end{javacode}

Luokkarakenteella voidaan vastata kaikkiin kohtiin a-d. Vastaukset käydään läpi
seuraavissa aliluvuissa.

\section{A-tehtävä}

\label{A-tehtävä}

Neliönmuotoinen Tetris-kappale voidaan luoda ja tulostaa seuraavasti:

\begin{javacode}
Piece square = new Piece(Shape.SQUARE);
System.out.println(square.toString());
//Tulostaa konsoliin:
//
// **
// **
//
\end{javacode}

Itse kappaleen datarakenteen muodostaminen tapahtuu Shape-luokassa, joka tietää
minkä muotoinen on neliö. Shape-luokasta saatu datarakenne tallennetaan
Piece-luokan olion muodoksi ja se voidaan tulostaa.

\section{B-tehtävä}

\label{B-tehtävä}

Sattumanvarainen kappale voidaan luoda kirjoittamalla rivi:

\begin{javacode}
Piece random = new Piece(Shape.values()[r.nextInt(Shape.values().length)]);
\end{javacode}

jossa r on tehtävään tarvittu Random-olio.

\section{C-tehtävä}

\label{C-tehtävä}

Kappale voidaan pyöräyttää käyttämällä Piece-olion turnCoclwise() -metodia:

\begin{javacode}
//luodaan olio
Piece pyramid = new Piece(Shape.ZIGZAG);

//tulostetaan se ennen pyöräytystä
System.out.println(pyramid.toString());

//pyöritellään neljä kertaa kappaletta
for ( int i = 0; i < 4; i++) {
	//Pyöräytetään 90° astetta myötäpäivään
	pyramid.turnClockwise();
	
	//tulostetaan jokaisen pyörähdyksen jälkeen
	System.out.println(pyramid.toString());
}
//Tuloste konsolissa:
//
//*  
//** 
// * 
//   
//
//   
// **
//** 
//   
//
//   
//*  
//** 
// * 
//
//   
//** 
//**  
//   
//
//*  
//** 
// * 
//  
\end{javacode}

Pyöräytysmetodi käsittelee boolean arraytä tarkastelemalla jokaista indeksiä
seuraavanlaisesti:

\begin{javacode}
	public void turnClockwise() {
		boolean[][] oldGrid = grid;
		grid = new boolean[gridSize][gridSize];
		for ( int i = 0; i < gridSize; i++) {
			for ( int j = 0; j < gridSize; j++) {
				grid[i][j] = oldGrid[gridSize-j-1][i];
			}
		}
	}
\end{javacode}

jossa jokainen kohta ruuduosta (grid) käydään ja kirjoitetaan soluun uusi oikea
käännetty arvo.

\section{D-tehtävä}

\label{D-tehtävä}

Testataan mitä käy kappaleelle, jos sitä pyöräytetään neljäkertaa myötäpäivään.
JUnit5-testimetodi on rakennettu seuraavanlaisesti:

\begin{javacode}
/**
 * Creates random shape, rotates it 4 times and checks if it creates the same string
 */
@RepeatedTest(value = 1000)
void test() {
	//d-kohta
	Random r = new Random();
	Shape random = Shape.values()[r.nextInt(4)];
	
	Piece original = new Piece(random);
	Piece rotated = new Piece(random);
	
	for( int i = 0; i < 4; i++) {
		rotated.turnClockwise();
	}
	
	assertEquals(rotated.toString(), original.toString());
}
\end{javacode}

Testin aikana muodostetaan sattumanvarainen muoto, josta rakennetaan kaksi
identtistä kappaletta. Kappaleita pyöritetään neljäkertaa turnClockwise()
-metodilla ja tämän jälkeen kappaleiden tulostumista verrataan.

Tulokseksi saadaan 1000 kertaa toistetulla toistokokeella:

\begin{javacode}
// Runs: 1000/1000	Errors: 0	Failures:	0
\end{javacode}

joka osoittaa, että toString()-metodit palauttavat identtisen arvon, jolloin
kappale tulostuu saman muotoisena 4 pyöräytyksen jälkeen. Voidaan olettaa, että
silloin kappaleen grid-muuttujan data on myös identtinen pyöräytetyn kappaleen
datan kanssa. 

Tekijänä Pasi Toivanen (517487)


