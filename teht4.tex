
\chapter{Tehtävä 4 \label{chap:Teht=0000E4v=0000E4-4}}

Tehtävän tarkoituksena on luoda lukitusmenetelmä OOMKit-ohjelmiin, joka estää kaiken näkyvyyden ohjelmaan kunnes ohjelman avaava koodi syötetään täysin oikein. Aikarajaa salasanan syöttöön ei ole.

\section{Lukitus-luokka (kohdat a ja b)}

\label{Lukitus-luokka}

Kohdassa a tuli määrittää luokka Lukitus, jolla voisi lukita OOMKit-ohjelman. Kohdassa b se taas tuli toteuttaa. Kirjaan molemmat kohdat tähän samaan osioon.

\subsubsection{Lukitus-luokka}
\label{Lukitus-luokka}
\begin{javacode}
/**
 * @author Santeri Loitomaa
 */
public class Lukitus {
  private ArrayList<Key> codeLukituskoodi;
  private ArrayList<Key> yritysLukituskoodi;
  private boolean lukossa;
  /**
   * Konstruktori lukolle.
   * @param codeLukituskoodi
   */
  public Lukitus(ArrayList<Key> codeLukituskoodi) {
    this.codeLukituskoodi = codeLukituskoodi;
    this.yritysLukituskoodi = new ArrayList<Key>();
    this.lukossa = false;
  }
  /**
   * Asettaa boolean lukossa-arvon oikeaksi.
   * @.pre Tätä metodia tulee kutsua vain jos ohjelma on lukossa.
   * @.post   if(merkki == oikein) lukossa = true;
   *       if(merkki != oikein) lukossa = true;
   *       (kun kaikki merkit oikein) lukossa = false;
   * @param Key merkki
   */
  public void tarkista(Key merkki){
    yritysLukituskoodi.add(merkki);
    if(codeLukituskoodi.equals(new ArrayList<Key>())) {
      yritysLukituskoodi = new ArrayList<Key>();
      lukossa = false;
    }
    else if(codeLukituskoodi.get(yritysLukituskoodi.size()-1) !=
        yritysLukituskoodi.get(yritysLukituskoodi.size()-1)) {
      System.out.println("Nyt meni väärin. Aloita alusta.");
      yritysLukituskoodi = new ArrayList<Key>();
      lukossa = true;
    }
    else if(codeLukituskoodi.size() == yritysLukituskoodi.size()) {
      System.out.println("Koodi on syötetty oikein.");
      yritysLukituskoodi = new ArrayList<Key>();
      lukossa = false;
    }
  }
  /**
   * Kertoo onko lukko lukossa.
   * @return true jos lukossa.
   */
  public boolean onLukossa() {
    return lukossa;
  }
  /**
   * Merkitsee lukon lukituksi.
   * @.post lukossa = true;
   */
  public void lukitse() {
    lukossa = true;
  }
  /**
   * Vaihtaa salasanan.
   * @.post codeLukituskoodi = codeLukituskoodi
   * @param codeLukituskoodi
   */
  public void setCodeLukituskoodi(ArrayList<Key> codeLukituskoodi) {
    this.codeLukituskoodi = codeLukituskoodi;
  }
  /**
   * Kertoo nykyisen lukituskoodin.
   * @return codeLukituskoodi
   */
  public ArrayList<Key> getCodeLukituskoodi(){
    return codeLukituskoodi;
  }
}
\end{javacode}

Tulin tällaisen lopputulokseen. Tämä toimii melko hyvin lisättynä OOMKitin DemoApp3:een pienten muutosten kera.

\pagebreak

\section{DemoApp3-luokka (kohta c)}

\label{DemoApp3-luokka}

DemoApp3:een tein seuraavat muutokset jotta Lukitus-luokkaa voitaisiin käyttää hyvin.

\begin{javacode}
class LaatikkoGameLogic2 implements AppLogic {
  ...
  private Lukitus lukko = new Lukitus(new ArrayList<Key>());
  private boolean vaihdetaankoSalasana = false;
  private boolean vaihdetaanSalasanaa = false;
  private ArrayList<Key> uusiSalasana = new ArrayList<Key>();
  private boolean lukitaanko = false;
  ...
  @Override
  public void tick() {
    ...
    // piilottaa näkymän jos lukossa
    piirtoPinta.asetaPiilotus(lukko.onLukossa());
    ...
  }
  // käsittele näppäimen painallus
  @Override
  public void handleKey(Key k) {
    System.out.println(k);
    if(lukko.onLukossa()) {
      lukko.tarkista(k);
    }
    else if(vaihdetaanSalasanaa) {
      uusiSalasana.add(k);
      if(k == Key.Backspace) {
        uusiSalasana = new ArrayList<Key>();
        System.out.println("Salasanan vaihto peruutettu.");
        vaihdetaanSalasanaa = false;
        vaihdetaankoSalasana = false;
      }
      if(k == Key.Enter) {
        lukko.setCodeLukituskoodi(uusiSalasana);
        uusiSalasana = new ArrayList<Key>();
        System.out.println("Salasana vaihdettu.");
        vaihdetaanSalasanaa = false;
        vaihdetaankoSalasana = false;
       }
    }
    else if(lukitaanko && k == Key.N) {
      lukitaanko = false;
      System.out.println("Lukitus peruutettiin.");
    }
    else if(lukitaanko && k == Key.Y) {
      lukitaanko = false;
      lukko.lukitse();
      System.out.println("Lukittu.");
    }
    else if(vaihdetaankoSalasana && k == Key.N) {
      vaihdetaankoSalasana = false;
    }
    else if(vaihdetaankoSalasana && k == Key.Y) {
      vaihdetaanSalasanaa = true;
      System.out.println("Anna uusi salasana merkki kerrallaan.");
      System.out.println("Tallenna painamalla Enter.");
      System.out.println("Peruuta painamalla Backspace.");
    }
    else if(k == Key.Space) {
      System.out.println("Haluatko lukita ohjelman? Y/N");
      lukitaanko = true;
    }
    else if(k == Key.Backspace) {
      System.out.println("Nykyinen salasana:");
      System.out.println(lukko.getCodeLukituskoodi());
    }
    else if(k == Key.Enter) {
      System.out.println("Haluatko asettaa uuden salasanan? Y/N");
      vaihdetaankoSalasana = true;
    }
  }
}
\end{javacode}

Lukitusta voidaan hallita handleKey(Key k)-metodilla.

\pagebreak

\section{Testi-luokka (kohta d)}

\label{Testi-luokka}

En juuri tiedä, miten testiluokka tulisi tehdä Jqwikilla, joten teen sen JUnitilla. Ensin testataan oikealla salasanalla ja sitten uudelleen lukittuna väärällä.


\begin{javacode}
class Testi {
  static Lukitus lukko;
  static ArrayList<Key> salasana = new ArrayList<Key>();
  @BeforeAll
  static void setUpBeforeClass() throws Exception {
    salasana.add(Key.T);
    salasana.add(Key.E);
    salasana.add(Key.S);
    salasana.add(Key.T);
    salasana.add(Key.I);
    lukko = new Lukitus(salasana);
    lukko.lukitse();
  }
  @Test
  void test() {
    lukko.tarkista(Key.T);
    lukko.tarkista(Key.E);
    lukko.tarkista(Key.S);
    lukko.tarkista(Key.T);
    lukko.tarkista(Key.I);
    if(lukko.onLukossa()) {
      fail("Salasana ei toiminut");
    }
    lukko.lukitse();
    lukko.tarkista(Key.T);
    lukko.tarkista(Key.E);
    lukko.tarkista(Key.S);
    lukko.tarkista(Key.Y);
    lukko.tarkista(Key.I);
    if(!lukko.onLukossa()) {
      fail("Salasana toimi vaikkei pitänyt");
    }
  }
}
\end{javacode}

\pagebreak 

Testi ei tuottanut yhtään virhettä ja konsolin tulosteet näyttävät oikealta.

\begin{javacode}
Koodi on syötetty oikein.
Nyt meni väärin. Aloita alusta.
Nyt meni väärin. Aloita alusta.
\end{javacode}

Tekijänä Santeri Loitomaa (516587)