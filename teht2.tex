
\chapter{Tehtävä 2 \label{chap:Teht=0000E4v=0000E4-2}}

\section{a-kohta}
\label{a-kohta}
Loin luokan Kuvio, jolla on aiemmassa tehtävässä määritellyt ominaisuudet väri, täytetty,
keskipiste ja lisäksi kulmien määrä, joka määrittää mikä kuvion tyyppi on kyseessä. Tyypit
olivat tehtävässä enumeina YMPYRÄ, NELIÖ, KOLMIO.
\begin{javacode}
public class Kuvio{
  int kulmaMaara;
  Point keskusta;
  Color vari;
  double koko;
  boolean taytetty;
  enum Tyyppi{
    NELIÖ, YMPYRÄ, KOLMIO
  }
  Tyyppi tyyppi;
  
  /**
   * 
   * Luo uuden kuvio olion, katkaisten luonnin jos kulmaMaara ei täsmää kuvioihin.
   * @.pre kulmaMaara = 0 || 3 || 4
   * @.post new Kuvio
   */
  public Kuvio(int kulmaMaara, Point keskusta, Color vari, boolean taytetty) {
    if (kulmaMaara != 0 && kulmaMaara != 3 && kulmaMaara != 4) {
      System.out.println("Kuvio ei ole ympyrä, neliö tai kolmio!");
      return;
    }
    this.kulmaMaara = kulmaMaara;
    this.keskusta = keskusta;
    this.vari = vari;
    this.taytetty = taytetty;
    if (kulmaMaara == 0) {
      this.tyyppi = Tyyppi.YMPYRÄ;
      this.koko = 100;
    }
    else if(kulmaMaara == 3){
      this.tyyppi = Tyyppi.KOLMIO;
      this.koko = 30;
    }
    else if(kulmaMaara == 4){
      this.tyyppi = Tyyppi.NELIÖ;
      this.koko = 60;
    }
  }
  
}

\end{javacode}
Kuviolla on etukäteen määritetty koko, jotta tehtävä täyttäisi suuruusjärjestykseen
liittyvät ehdot.

Kuviopari perii luokan kuvio ja toimii hyvin samalla tavalla, mutta sillä on lisää
muuttujia sisemmän kuvion ominaisuuksille. Sisemmän ja ulomman kuvion kokoja muokataan
tässä sen mukaan onko sisäkuvio sama vai eri.
\begin{javacode}
public class KuvioPari extends Kuvio{
  int sisaKulmaMaara;
  Color sisaVari;
  double sisaKoko;
  boolean sisaTaytetty;
  enum Tyyppi{
    NELIÖ, YMPYRÄ, KOLMIO
  }
  Tyyppi sisaTyyppi;
  
  /**
   * 
   * Luo uuden kuviopari olion, katkaisten luonnin jos kulmaMaara ei täsmää kuvioihin.
   * @.pre kulmaMaara = 0 || 3 || 4 && sisaKulmaMaara = 0 || 3 || 4
   * @.post new KuvioPari
   */
  public KuvioPari(int kulmaMaara, int sisaKulmaMaara, Point keskusta, Color vari,
  Color sisaVari, boolean taytetty, boolean sisaTaytetty) {
    super(kulmaMaara,keskusta,vari,taytetty);
    if (sisaKulmaMaara != 0 && sisaKulmaMaara != 3 && sisaKulmaMaara != 4) {
      System.out.println("Sisäkuvio ei ole ympyrä, neliö tai kolmio!");
      return;
    }
    this.sisaKulmaMaara = sisaKulmaMaara;
    this.sisaVari = sisaVari;
    this.sisaTaytetty = sisaTaytetty;
    if (sisaKulmaMaara == 0) {
      this.sisaTyyppi = Tyyppi.YMPYRÄ;
      this.sisaKoko = 100;
    }
    else if(kulmaMaara == 3){
      this.sisaTyyppi = Tyyppi.KOLMIO;
      this.sisaKoko = 30;
    }
    else if(kulmaMaara == 4){
      this.sisaTyyppi = Tyyppi.NELIÖ;
      this.sisaKoko = 60;
    }
    if (kulmaMaara == sisaKulmaMaara) {
      this.koko = this.koko + 20;
      this.sisaKoko = this.sisaKoko + 10;
    }else {
      this.koko = this.koko + 10;
      this.sisaKoko = this.sisaKoko - 10;
    }
  }
}
\end{javacode}

Koska kuviopari perii kuvion, molempia voi käyttää Kuvion metodissa vertaile,
joka laskee kuvioiden pinta-alan ja määrittelee siten mikä on suurin.
\begin{javacode}
  /**
   * Vertailee kahta kuviota toisiinsa koon suhteen
   * @.post System.out.println("Toinen kuvio on isompi." || "Ensimmäinen kuvio on
   isompi." || "Kuviot ovat yhtä suuret");
   */
  public void vertaile(Kuvio kuvio) {
    double suuruus1 = 0;
    double suuruus2 = 0;
    if (this.tyyppi == Tyyppi.KOLMIO) {
      suuruus1 = this.koko*this.koko*0.433;
    }
    if (this.tyyppi == Tyyppi.YMPYRÄ) {
      suuruus1 = (this.koko/2)*(this.koko/2)*Math.PI;
    }
    if (this.tyyppi == Tyyppi.NELIÖ) {
      suuruus1 = this.koko*this.koko;
    }
    if (kuvio.tyyppi == Tyyppi.KOLMIO) {
      suuruus2 = kuvio.koko*kuvio.koko*0.433;
    }
    if (kuvio.tyyppi == Tyyppi.YMPYRÄ) {
      suuruus2 = (kuvio.koko/2)*(kuvio.koko/2)*Math.PI;
    }
    if (kuvio.tyyppi == Tyyppi.NELIÖ) {
      suuruus2 = kuvio.koko*kuvio.koko;
    }
    
    if (suuruus1 < suuruus2) {
      System.out.println("Toinen kuvio on isompi.");
    } else if(suuruus1 > suuruus2) {
      System.out.println("Ensimmäinen kuvio on isompi.");
    } else {
      System.out.println("Kuviot ovat yhtä suuret");
    }
  }
  
\end{javacode}

Lopuksi testasin vertailun toimivuutta luomalla kuviot testi1-9 jotka ovat kukin
joko pelkkä kuvio, kuvio joka sisältää eri kuvion kuin itsensä ja kuvio joka sisältää
itsensä muotoisen kuvion. Testasin aluksi vartavasten test2 ja test1 kokoaeroa ja
sitten test2 itsensä kanssa varmistaakseni vertailumetodin antavan kaikki halutut
tulokset. Sitten testasin loput järjestyksessä, varmistaakseni niiden järjestyksen
olevan tehtävänannon ohjeiden mukainen.

Alla kutsupino ja sitten tulostus
\begin{javacode}
      Point keski = new Point(100, 100);
      Kuvio test1 = new Kuvio(3,keski,CoreColor.Red,true);
      KuvioPari test2 = new KuvioPari(3,4,keski,CoreColor.Red,CoreColor.Red,true,true);
      KuvioPari test3 = new KuvioPari(3,3,keski,CoreColor.Red,CoreColor.Red,true,true);
      Kuvio test4 = new Kuvio(4,keski,CoreColor.Red,true);
      KuvioPari test5 = new KuvioPari(4,0,keski,CoreColor.Red,CoreColor.Red,true,true);
      KuvioPari test6 = new KuvioPari(4,4,keski,CoreColor.Red,CoreColor.Red,true,true);
      Kuvio test7 = new Kuvio(0,keski,CoreColor.Red,true);
      KuvioPari test8 = new KuvioPari(0,4,keski,CoreColor.Red,CoreColor.Red,true,true);
      KuvioPari test9 = new KuvioPari(0,0,keski,CoreColor.Red,CoreColor.Red,true,true);
      test2.vertaile(test1);
      test2.vertaile(test2);
      test2.vertaile(test3);
      test3.vertaile(test4);
      test4.vertaile(test5);
      test5.vertaile(test6);
      test6.vertaile(test7);
      test7.vertaile(test8);
      test8.vertaile(test9);
      
''Ensimmäinen kuvio on isompi.
  Kuviot ovat yhtä suuret
  Toinen kuvio on isompi.
  Toinen kuvio on isompi.
  Toinen kuvio on isompi.
  Toinen kuvio on isompi.
  Toinen kuvio on isompi.
  Toinen kuvio on isompi.
  Toinen kuvio on isompi.''
\end{javacode}

\section{b-kohta}
\label{b-kohta}
Kuvio voi olla vain ympurä, neliö tai kolmio, sillä konstruktori varmistaa
kulmamäärän heti alussa ja katkaisee koodin toiminnan jos lukema ei täsmää.
\begin{javacode}
    if (kulmaMaara != 0 && kulmaMaara != 3 && kulmaMaara != 4) {
      System.out.println("Kuvio ei ole ympyrä, neliö tai kolmio!");
      return;
    }
\end{javacode}

\section{c-kohta}
\label{c-kohta}
Kaikki luokat toimivat ArrayListissa, HashMapissa ja TreeMapissa, niin kauan kun tiesi
joko objektin indexin listassa tai avaimen jolla sen sai haettua, sillä silloin ne sai
ulos alkuperäisessä muodossaan ja pystyi castaamaan tarvittaessa. HashSetissä ja
TreeSetissä, ei objekteja saanut palautettua samalla tavalla, joten mikään ei toiminut.

\section{d-kohta}
\label{d-kohta}
Luokassa Tallennettava on kaksi Serializable arvoa jotka tallennetaan, ja yksi transient
arvo jota ei tallenneta, mainissa oleva tallennusmetodi ajettaessa syntyi tallennus.tal
tiedosto.
\begin{javacode}
import java.io.FileOutputStream;
import java.io.IOException;
import java.io.ObjectOutputStream;
import java.io.Serializable;

public class Main {
  public static void main(String[] args) {
    Tallennettava t = new Tallennettava();
    try {
      FileOutputStream tiedosto = new FileOutputStream("tallennus.tal");
      ObjectOutputStream tallenna = new ObjectOutputStream(tiedosto);
      tallenna.writeObject(t);
      tallenna.close();
    } catch (IOException e) {
      e.printStackTrace();
    }
  }
  
}
class Tallennettava implements Serializable{
  private static final long serialVersionUID = 1L;
  int arvo1 = 24;
  transient int arvo2 = 10;
  int arvo3 = 12;
}
\end{javacode}


Tekijänä Tommi Heikkinen (517749)
