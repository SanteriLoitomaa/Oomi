
\chapter{Tehtävä 2 \label{chap:Teht=0000E4v=0000E4-2}}

\section{a-kohta}
\label{a-kohta}
Määrittelyn muutoksen aloittaisin metodista lisääSolmu, koska solmuja ei enää
luoda irrallisina vaan osana muita prosesseja, voi sen tyypin vaihtaa privateksi.
\begin{javacode}
/**
* @.pre leima != null && !sisältääSolmun(leima)
* @.post RESULT.sisältääSolmun(leima) &
* RESULT.poistaSolmu(leima).equals(this)
*/
private lisääSolmu(String leima)
\end{javacode}
Vaihtoehtoisesti tyypin voi pitää publicina, jos metodia kuitenkin haluaa luokan
ulkopuolelta kutsua.

Tärkeämpi osuus on muuttaa SuunnattuGraafi luokan constructoria siten, että
uutta graafia luodessa pitää sille antaa syötteenä luotava solmu, josta
rakentaminen lähtee liikkeelle (ns. juurisolmu).

\begin{javacode}
/**
* @.pre juuri != null
* @.post solmumäärä() == 1 & kaarimäärä() == 0 &
*        RESULT.sisältääSolmun(juuri) &
*        RESULT.poistaSolmu(juuri).equals(this)
*/
public SuunnattuGraafi(String juuri)
\end{javacode}
Konstruktori siis kutsuu aiemmin muokattua lisääSolmu(leima) metodia ja luo
sen pohjalta juurisolmun.

Uuden kaaren lisäämisessä pitää huomioida, että kohdejuuren ei ole pakko olla vielä
olemassa, vaan se voidaan luoda kaaren yhteydessä, mutta muita muutoksia metodi ei
juuri vaadi.
\begin{javacode}
/**
* @.pre (lähtöleima != null & tuloleima != null) &&
* sisältääSolmun(lähtöleima) & !sisältääKaaren(lähtöleima, tuloleima)
* @.post RESULT.sisältääKaaren(lähtöleima, tuloleima) &
* RESULT.poista(lähtöleima, tuloleima).equals(this) &
* RESULT.annaPaino(lähtöleima, tuloleima) == paino &
* if(!sisältääSolmun(tuloleima))  RESULT.sisältääSolmun(tuloleima) &
*                                 RESULT.poistaSolmu(tuloleima).equals(this)
*/
public SuunnattuGraafi lisääKaari(String lähtöleima, String tuloleima,
double paino)
\end{javacode}
Jos tuloleima nimistä solmua ei ole vielä olemassa metodi luo sellaisen, jatkaen
sitten toimintaansa normaalisti. Jos solmu on jo olemassa toiminta ei eroa aiemmasta.


\section{b-kohta}
\label{b-kohta}
Luokan SuunnattuGraafi nimi kannattaisi alkajaisiksi muuttaa, esim. SuunnatonGraafi,
vastaavasti kaikki metodit jotka viittaavat tähän nimeen yms. tulisi muuttaa viitaamaan
uuteen nimeen. Terminologian selkeyttämiseksi vaihtaisin myös muuttujat lähtöleima ja
tuloleima, vaikkapa korvaaviksi päätysolmu1 ja päätysolmu2, kuvastamaan ettei ole alkua
ja loppua vaan kaaren kaksi päätyä.

Terminologian muutosten lisäksi on luokassa vain kaksi metodia jotka ovat kiinnostuneita
kaaren edeltäjist ja seuraajista, metodit Solmujoukko edeltjät & Solmujoukko seuraajat.
Nämä voisi poistaa ja korvata uudella versiolla Solmujoukko parit
\begin{javacode}
/**
* @.pre leima != null && sisältääSolmun(leima)
* @.post RESULT == (solmuun linkitettyjen kaarien toisissa päissä olevat solmut)
*/
public Solmujoukko parit(String leima)
\end{javacode}
Solmujoukko parit palauttaa kaikki solmut, jotka ovat yhteydessä kyseiseen slomuun jo
olemassaolevien kaarien kautta.


Tekijänä Tommi Heikkinen (517749)
