
\chapter{Tehtävä 2 \label{chap:Teht=0000E4v=0000E4-2}}

\section{a-kohta}
\label{a-kohta}
Loin data-abstraktioksi luokan Sudoku, jolla on ominaisuus kenttä. Kenttä on 9x9
matriisi, joka automaattisesti sisältää int arvoja välillä 0-9 (defaultilla 0).
Alla Sudoku luokan constructori.
\begin{javacode}
/**
   * Luo sudokuruudukko olion, jolla on ominaisuutena 9x9 matriisi kenttä.
   * @.pre true
   * @.post new Sudoku
   */
  public Sudoku(){
    int[][] kenttä = new int [9][9];
  }
\end{javacode}
Luokka sisältää myös metodit lukujen katsomiseen tietyssä ruudussa ja lukujen lisäämiseen.


\section{b-kohta}
\label{b-kohta}
 
 metodi onSudoku() ottaa vastaan objektin ja kertoo sitten oikeellisesti onko kyseessä sudoku olio, koska olion
 kenttä on jo valmiiksi määritelty siten ettei se ota vastaan muuta kuin inttejä, niin kauan kun olio on sudoku
 on sen data myös oikeellinen.
\begin{javacode}
/**
   * Tarkistaa että onko annettu objekti sudoku-data
   * @.pre true
   * @.post return true); || return false;
   */
  public static boolean onSudoku(Object o){
      if (o instanceof Sudoku) {
        return true;
      }else {
        return false;
      }
  }
\end{javacode}


\section{c-kohta}
\label{c-kohta}

Metodi sudokuTila aloittaa tarkistamalla onko kyseesäs sudoku, hypäten suoraan loppuun jo se ei ole. Metodi käy
sitten systemaattisesti läpi eri vaihtoehdot, tarkistaen onko jokaisessa ruudussa eri luku kuin 0. Tutkien sitten
jokaisen pyst ja vaakarivin ja ruudukon, varmistaen ettei sama luku esiinny kahdesti. Kaikista näistä printataan
teksti joka kertoo lopputuloksen.
\begin{javacode}
/**
   * Tarkistaa onko annettu objekti sudoku dataa, kertoo sen
     jälkeen onko ruudukko oikein täytetty ja täysi.
   * @.pre true
   * @.post (System.out.println("Objekti on Sudoku"); ||
     System.out.println("Objekti ei ole Sudoku");) &&
     ( System.out.println("Sudoku on väärin täytetty"); ||
     System.out.println("Sudoku on oikein täytetty");) &&
     (System.out.println("Sudoku on täysi"); ||
     System.out.println("Sudoku ei ole täysi"); )
   */
  public static void sudokuTila(Object o){
    boolean täysi = true;
    boolean oikein = true;
    boolean brake = false;
    if (o instanceof Sudoku) {
        System.out.println("Objekti on Sudoku");
        // tästä alkaa tarkistus onko sudoku
         täytetty vai kesken
        for (int x=0; x<9; x++) {
          for (int y=0; y<9; y++) {
            if (((Sudoku) o).kenttä[x][y]==0) {
              täysi = false;
            }
          }
        }
        //Tästä alkaa tarkistus onko sudoku oikein
        täytetty (vasta vaakarivit)
        for (int luku=1;luku<10;luku++) {
          for (int x=0; x<9; x++) {
            int toistox = 0;
            int toistoy = 0;
            for (int y=0; y<9; y++) {
              if (((Sudoku) o).kenttä[x][y]==luku) {
                toistox ++;
              }
              if (((Sudoku) o).kenttä[y][x]==luku) {
                toistoy ++;
              }
            }
            if (brake){
              break;
            }
            if (toistox>1 || toistoy>1) {
              System.out.println("Sudoku on väärin
              täytetty");
              brake = true;
            }
          }
          if (brake){
            break;
          }
          for (int a=0;a<3;a++) {
            for (int b=0;b<3;b++) {
              int toistoz = 0;
              for (int x=0;x<3;x++) {
                for (int y=0;y<3;y++) {
                  if (((Sudoku) o).kenttä[x+a*3]
                  [y+b*3]==luku) {
                    toistoz ++;
                  }
                  if (toistoz>1) {
                    System.out.println("Sudoku
                    on väärin täytetty");
                    brake = true;
                  }
                  if (brake){
                    break;
                  }
                }
                if (brake){
                  break;
                }
               }
              if (brake){
                break;
              }
            }
            if (brake){
              break;
            }
          }
          if (brake){
            break;
          }
        }
        if (brake == false) {
          System.out.println("Sudoku on oikein
          täytetty");
        }
        if (täysi) {
          System.out.println("Sudoku on täysi");
        }else {
          System.out.println("Sudoku ei ole täysi");
        }
      }else {
        System.out.println("Objekti ei ole Sudoku");
      }
    }
\end{javacode}
Vaihtoehtoisesti luokka voisi sisältää booleanit täysi, oikein ratkaistu yms. ja printtaukset voisi silloin
korvata booleanien muokkauksella.

\section{d-kohta}
\label{d-kohta}

Metodi sudokuKaanto, ottaa vastaan objekti, varmistaen aluksi onko kyseessä sudoku, sitten kääntää muuttujan kenttä
lukujen paikkoja. Tarkentaisin tehtävänantoa määrityksellä siitä kumpaan suuntaan sudokua tulee kääntää.
(tämä versio kääntää myötäpäivään.)
\begin{javacode}
  /**
   * Siirtää sudokun muuttujan kenttä lukujen paikkaa niin kuin
     oliota käännettäisiin 90 astetta.
   * @.pre true
   * @.post s.kenttä[x][y] --> s.kenttä[8-y][x] ||
     System.out.println("Objekti ei ole sudoku!");
   */
  public static void sudokuKaanto(Object s){
    if(onSudoku(s)) {
      Sudoku clone = new Sudoku();
      for (int x=0;x<9;x++) {
        for (int y=0;y<9;y++) {
          clone.täytä(8-y,x,((Sudoku)s).kenttä[x][y]);
        }
      }
      for (int x=0;x<9;x++) {
        for (int y=0;y<9;y++) {
          ((Sudoku)s).täytä(x,y,clone.kenttä[x][y]);
        }
      }
    }else {
      System.out.println("Objekti ei ole sudoku!");
    }
  }
\end{javacode}

\section{e-kohta}
\label{e-kohta}

Testin onnistuminen on varma, koska sudokun kääntö ei muuta sudoku oliota toiseksi olioksi, vaan muuttaa lukuja sen
muuttujassa.

\begin{javacode}
class FlipTest {

  @RepeatedTest(value = 1000)
  void test() {
    Sudoku sudoku = new Sudoku();
    T2.sudokuKaanto(sudoku);
    assertTrue(T2.onSudoku(sudoku));
  }

}
\end{javacode}
Lopputulos 0 errors & 0 failures

Tekijänä Tommi Heikkinen (517749)
