
\chapter{Tehtävä 2 \label{chap:Teht=0000E4v=0000E4-2}}

\section{a-kohta}
\label{a-kohta}
Loin uuden MatoPelaaja luokan, jonne tallentuu pelaajan pisteet kokonaislukuna, nimi lukujonona,väri javafx Color-oliona ja sen hetkinen pituus myös kokonaislukuna. Luokasta löytyy metodit pisteiden ja pituuden muuttamiseen.
\section{a-kohta}
\label{a-kohta}
Loin myös MatoNäkymä luokan jonka avulla mato piiretään peliin. Luokassa on madon
pää erikseen pisteenä ja loppu mato ArrayListissa ja tietää nykyisen suunnan.

Käytin tehtävässä pohjana valmista pohjakoodia jota olen muutellut jonkin verran.

\begin{javacode}
package fi.utu.oomkit.examples.emptyapp;



import fi.utu.oomkit.AppConfiguration;
import fi.utu.oomkit.AppLogic;
import fi.utu.oomkit.gui.MainWindow;
import fi.utu.oomkit.gui.OOMApp;
import fi.utu.oomkit.gui.SimpleCanvas;
import fi.utu.oomkit.util.Point;
import fi.utu.oomkit.util.KeyHandler.Key;
import javafx.scene.canvas.GraphicsContext;
import javafx.scene.input.KeyEvent;
import java.awt.event.*;

/**
 * LaatikkoPiirtopinta. Käyttöliittymä huolehtii teknisestä toteutuksesta.
 * @see this.redraw() Piirrä uudestaan, riittää kutsua vain kun sisältö muuttuu.
 * @see this.drawForegroundContent(canvas) piirtää "edustan"
 * @see this.drawBackgroundContent(canvas) piirtää "taustan", oletus: täysin valkoinen
 */
class LaatikkoPiirtopinta extends SimpleCanvas {
    private final int koko = 12;
    private Point piste;
    private MatoNäkymä mauno;

    LaatikkoPiirtopinta() { super(true);
    mauno=new MatoNäkymä();}

    // asettaa piirrettävän pisteen
    void asetaPiste(Point piste) {
        this.piste = piste;
    }

    // piirtää pisteen p size-kokoisena neliönä, vas. yläreuna = p
    private void drawRectangle(GraphicsContext canvas, Point p, int size) {
        canvas.setFill(javafx.scene.paint.Color.DARKBLUE);
        canvas.fillRect(p.x, p.y, size, size);
        
        
    }

    // piirrä piste, jos koordinaatti olemassa (piste != null)
    protected void drawForegroundContent(GraphicsContext canvas) {
        if (piste != null) {
        	drawRectangle(canvas, piste, koko);
        	}
        if(mauno!=null) {
        	mauno.draw(canvas);
        }
    }
}

class LaatikkoGameLogic implements AppLogic {
    final LaatikkoPiirtopinta piirtoPinta = new LaatikkoPiirtopinta();
    private final int reuna = 25;

    private int liikkeenTila = 0;
    private Point piste = new Point(reuna, reuna);
    private MatoNäkymä mauno = new MatoNäkymä();
    Point pää = mauno.getPää();
  

    // alustaa pelin logiikan, päivitys 20ms välein
    @Override
    public AppConfiguration configuration() {
        return new AppConfiguration(20, "Demo", false);
    }

    @Override
    public void tick() {
        double max_x = piirtoPinta.getWidth() - reuna;
        double max_y = piirtoPinta.getHeight() - reuna;
        switch (mauno.AnnaSuunta()) {
       
            case 'D':
            	
                if (piste.x < max_x) {
                	piste = piste.add(5, 0);
                }
                
                
                
                break;
            case 'S':
                if (piste.y < max_y)
                    piste = piste.add(0, 5);
                
                break;
            case 'A':
                if (piste.x > reuna)
                    piste = piste.add(-5, 0);
                
                break;
            case 'W':
            	
                if (piste.y > reuna)
                    piste = piste.add(0, -5);
                break;
        }
        
        piirtoPinta.asetaPiste(mauno.getPää());
        piirtoPinta.asetaPiste(piste);
        piirtoPinta.redraw();
    }

public void Törmäys() {
	if(piste.x==mauno.getPää().x && piste.y==mauno.getPää().y) {
		piirtoPinta.asetaPiste(Point.generateRandomPoint((int)piirtoPinta.getWidth(),(int)piirtoPinta.getHeight()));
	}
	if()
}

public void handleKey(Key k) {
	if(k.equals(Key.A)) {
		piste = piste.add(-5,0);
		mauno.vaihdaSuunta('A');
	}
	if(k.equals(Key.W)) {
		piste = piste.add(0,-5);
		mauno.vaihdaSuunta('W');
	}
	if(k.equals(Key.D)) {
		piste = piste.add(5,0);
		mauno.vaihdaSuunta('D');
	}
	if(k.equals(Key.S)) {
		piste = piste.add(0,5);
		mauno.vaihdaSuunta('S');
	}
	System.out.println(k + " pressed!"); 
}
}


public class DemoApp1 extends OOMApp {
    // alustaa pelilogiikan
    final static LaatikkoGameLogic gameLogic = new LaatikkoGameLogic();

    // kytkee piirtopinnan käyttöliittymään
    @Override
    protected MainWindow generateMainWindow(String appName, double width, double height) {
        return new SimpleMainWindow(appName, width, height) {
            @Override public SimpleCanvas mainContent() {
                return gameLogic.piirtoPinta;
            }
        };
    }

    // alustaa demo-ohjelman (käyttöliittymä)
    public DemoApp1() {
        super(gameLogic);
    }

    // alustaa demo-ohjelman (JavaFX)
    public static void main(String[] args) {
        launch(args);
    }

}
\end{javacode}

\section{c-kohta}
\label{c-kohta}
MatoPelaajan koodi.
\begin{javacode}
package fi.utu.oomkit.examples.emptyapp;

import javafx.scene.paint.Color;

public class MatoPelaaja {
	
	private int pistesumma;
	private String nimi;
	 Color väri;
	private int pituus ;
	
	public MatoPelaaja(String nimi) {
		this.nimi=nimi;
		pituus = 5;
		väri = Color.GOLD;
	}
	public void addPoints(int p) {
		pistesumma = pistesumma+p;
	}
	public void muutaPituus(int p) {
		pituus += p;
	}

}

\end{javacode}

\section{d-kohta}
\label{d-kohta}

MatoNäkymän Koodi.

\begin{javacode}
package fi.utu.oomkit.examples.emptyapp;

import java.awt.event.KeyEvent;
import java.awt.event.KeyListener;
import java.util.ArrayList;

import fi.utu.oomkit.util.KeyHandler;
import fi.utu.oomkit.util.Point;
import javafx.scene.canvas.GraphicsContext;
import fi.utu.oomkit.util.KeyHandler.Key;

//Luokka Madon piirtämisen avuksi
public class MatoNäkymä   {
	private Point pää;
	ArrayList<Point> häntä;
	protected char suunta;
	protected MatoPelaaja mauno = new MatoPelaaja("Mauno Matonen");
	
	public MatoNäkymä() {
		pää = new Point(100,25);
		suunta = 'D';
		häntä = new ArrayList<Point>();
		while(häntä.size()<5) {
			häntä.add(new Point(100-15*häntä.size(),25));}
		
	}
	public boolean törmäys(Point p) {
		if(pää.equals(p)) {

			int x = häntä.get(häntä.size()-1).x;
			int y = häntä.get(häntä.size()-1).y;
			häntä.add(new Point(x-15,y-15));
			return true;
		}else if(häntä.contains(p)) {return true;}
		else if(häntä.contains(pää)) {return true;}
		return false;
		}
	public Point getPää() {
		return pää;
	}
	public void setPää(Point pää) {
		this.pää=pää;
	}
	public void vaihdaSuunta(char c) {
		suunta = c;
	}
	public char AnnaSuunta() {
		return suunta;
	}
	
	public void draw(GraphicsContext canvas) {
		canvas.setFill(mauno.väri);
        canvas.fillRect(pää.x, pää.y, 12, 12);
        for(Point p : häntä) {
        	canvas.setFill(mauno.väri);
            canvas.fillRect(p.x, p.y, 12, 12);
        }
	}
	 
		
    
}


\end{javacode}







Tekijänä Janina Kuosmanen (516580)
