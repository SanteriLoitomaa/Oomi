
\chapter{Tehtävä 2 \label{chap:Teht=0000E4v=0000E4-2}}

\section{a-kohta}
\label{a-kohta}
Alla valmiskoodista otettu ympyrän piirämismetodi ja sen määritys.
\begin{javacode}
/**
* Piirtää tietyn värisen joko kokonaan väritetyn tai onton ympyrän, siten että annettu piste on keskipiste.
* @.pre koko>=0.0
* @.post Piirtää ympyrän annetulla värillä, koolla, sijainnilla (keskipisteenä) ja joko kokonaan väritettynä tai
* pelkillä reunoilla.
*/
  public void piirräPallukka(Point p, Color väri, double koko, boolean täytetty, GraphicsContext canvas) {
        if (täytetty){
            if (debuggaus) System.out.println("Piirretään täytetty " + (koko < 15 ? "pieni pallukka" : (int)koko+" cm - säteinen ympyrä") + " pisteeseen " + p);
            setFill(väri);
            canvas.fillOval(p.x - koko / 2, p.y - koko / 2, koko, koko);
        } else {
            if (debuggaus) System.out.println("Piirretään ontto " + (koko < 15 ? "pieni pallukka" : (int)koko+" cm - säteinen ympyrä") + " pisteeseen " + p);
            canvas.setLineWidth(2);
            setStroke(väri);
            canvas.strokeOval(p.x - koko / 2, p.y - koko / 2, koko, koko);
        }
    }
\end{javacode}

Alla valmiskoodista otettu neliön piirtämismetodi ja sen määrittely.
Muutin sen toimintaa siten että se ottaa vastaan keskipisteen ylä ja alakulman sijasta
ja koon, niinkuin tehtävänannossa vaadittiin.
\begin{javacode}
/**
* Piirtää tietyn värisen joko kokonaan väritetyn tai onton neliön, siten että annettu piste on keskipiste.
* @.pre koko>=0.0
* @.post Piirtää neliön annetulla värillä, koolla, sijainnilla (keskipisteenä) ja joko kokonaan väritettynä tai
* pelkillä reunoilla.
*/
    public void piirräLaatikko(Point p, Double koko, Color väri, boolean täytetty, GraphicsContext canvas) {
        if (täytetty) {
            if (debuggaus) System.out.println("Piirretään täytetty laatikko pisteiden " + (int)(p.x - koko/2) + "," + (int)(p.y - koko/2) + " ja " + (int)(p.x + koko/2) + "," + (int)(p.y + koko/2) + " välille");
            setFill(väri);
            canvas.fillRect(p.x - koko/2, p.y - koko/2, koko, koko);
        } else {
            if (debuggaus) System.out.println("Piirretään ontto laatikko pisteiden " + (int)(p.x - koko/2) + "," + (int)(p.y - koko/2) + " ja " + (int)(p.x + koko/2) + "," + (int)(p.y + koko/2) + " välille");
            canvas.setLineWidth(2);
            setStroke(väri);
            canvas.strokeRect(p.x - koko/2, p.y - koko/2, koko, koko);
        }
    }
\end{javacode}

Alla valmiskoodista otettu kolmionpiirtämismetodi ja sen määrittely, koodi ei täyttänyt
tehtävänannon vaatimuksia joten päivitin sen ottamaan vastaan etukäteen määritetyn värin ja
tiedon siitä onko se ontto voi väritetty.
\begin{javacode}
/**
* Piirtää tietyn värisen joko kokonaan väritetyn tai onton kolmion, siten että annettu piste on keskipiste.
* @.pre koko>=0.0
* @.post Piirtää kolmion annetulla värillä, koolla, sijainnilla (keskipisteenä) ja joko kokonaan väritettynä tai
* pelkillä reunoilla.
*/
    public void piirräKolmio(Point keski, Color väri, boolean täytetty, double koko, GraphicsContext canvas) {
        Point piste1 = kaarelta(keski, Math.PI * 1 / 6, koko / 2 / 37 * 50);
        Point piste2 = kaarelta(keski, Math.PI * 5 / 6, koko / 2 / 37 * 50);
        Point piste3 = kaarelta(keski, Math.PI * 9 / 6, koko / 2 / 37 * 50);

        int siirtymäY = keski.y - (piste2.y + piste3.y) / 2;
        piste1 = piste1.add(0, siirtymäY);
        piste2 = piste2.add(0, siirtymäY);
        piste3 = piste3.add(0, siirtymäY);
    
    if (täytetty){
      setFill(väri);
      canvas.fillPolygon(new double[]{piste1.x, piste2.x, piste3.x}, new double[]{piste1.y, piste2.y, piste3.y}, 3);
    } else{
      canvas.setLineWidth(2);
            setStroke(väri);
            canvas.strokePolygon(new double[]{piste1.x, piste2.x, piste3.x}, new double[]{piste1.y, piste2.y, piste3.y}, 3);
    }
        
    }
\end{javacode}


\section{b-kohta}
\label{b-kohta}
Perusasetelman kuviot saadaan aikaan sijoittamalla piirrä-metodit piirtelypinann drawforegroundin sisään.
\begin{javacode}
    protected void drawForegroundContent(GraphicsContext canvas) {

        {
            Point keski = new Point(100, 100);
            double koko = 32;
            piirräLaatikko(keski, koko, CoreColor.Red, true, canvas);
        }
        {
            Point keski = new Point(300, 150);
            double koko = 80;
            piirräKolmio(keski, CoreColor.Green, true, koko, canvas);
        }
        {
            Point keski = new Point(150, 350);
            int koko = 80; 
            piirräPallukka(keski, CoreColor.Yellow, koko, true, canvas);
        }
    }
\end{javacode}

\section{c-kohta}
\label{c-kohta}
Asetelma laittaa sisäkkäiset kuviot, joista sisempi on puolet pienempi ennalta määrättyihin
sijainteihin. Kunkin kuvion lokaatio määritellään randomilla, hyväksikäyttäe switchia, varmistaen
ettei sama ole missään kahdesti.

\begin{javacode}
Random num = new Random();
      int x = 0;
      int y = 0;
      double koko = 30;
      int kohta = num.nextInt(9);
      ArrayList käytetyt = new ArrayList();
      CoreColor color = CoreColor.Red;
      for (int i = 0; i<3;i++) {
        x = x + 50;
        for (int j = 0; j<3;j++) {
          y = y + 50;
          Point p = new Point(x,y);
          while (käytetyt.contains(kohta)) {
            kohta = num.nextInt(9);
          }
          käytetyt.add(kohta);
          switch(kohta){
          case 0: piirräKolmio(p, color, false, koko, canvas);
          piirräKolmio(p, color, true, koko/2, canvas);
          break;
          case 1: piirräKolmio(p, color, false, koko, canvas);
          piirräPallukka(p, color, koko/2, true, canvas);
          break;
          case 2: piirräKolmio(p, color, false, koko, canvas);
          piirräLaatikko(p, koko/2, color, true, canvas);
          break;
          case 3: piirräPallukka(p, color, koko, false, canvas);
          piirräKolmio(p, color, true, koko/2, canvas);
          break;
          case 4: piirräPallukka(p, color, koko, false, canvas);
          piirräPallukka(p, color, koko/2, true, canvas);
          break;
          case 5: piirräPallukka(p, color, koko, false, canvas);
          piirräLaatikko(p, koko/2, color, true, canvas);
          break;
          case 6: piirräLaatikko(p, koko, color, false, canvas);
          piirräKolmio(p, color, true, koko/2, canvas);
          break;
          case 7: piirräLaatikko(p, koko, color, false, canvas);
          piirräPallukka(p, color, koko/2, true, canvas);
          break;
          case 8: piirräLaatikko(p, koko, color, false, canvas);
          piirräLaatikko(p, koko/2, color, true, canvas);
          break;
        }
        }
        y = 0;
      }
\end{javacode}

Ratkaisussa ei tarvinnut käyttää periytymistä aiempia enempää, joskin PiirtelyPinta, jolla kuviot
piirretään perii luokan SimpleCanvas.

\section{d-kohta}
\label{d-kohta}
Yksi tapa ratkaista tilanne on luoda uudet metodit KolmioNeliö ja Neliökolmio yms ja laittaa ne
kutsumaan jo olemassaolevia metodeja, jolloin voidaan hyväksikäyttää jo olemassaolevia materiaaleja.
esim.
\begin{javacode}
KolmioNeliö(Point p, double koko, Color color1, Color color2, boolean filled1, boolean filled2, GraphicsContext canvas)
\end{javacode}
Elegantimpi ratkaisu olisi uudenmallinen metodi, kenties jo olemassaolevista muokattu tai uusi, piirräMuoto,
joka ottaisi vastaan halutun muodon (tyypit tulisi olla määritetty etukäteen), kaikki muodolle tarpeelliset tiedot
ja myös mahdollisen sisämuodon tiedot, jotka nulleiksi jätettyinä tarkoittaisivat ettei sisämuotoa tule.


Tekijänä Tommi Heikkinen (517749)
